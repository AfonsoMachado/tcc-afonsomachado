%---------- Primeiro Capitulo ----------
\chapter{Introdução}

Este trabalho tem como objetivo investigar as noções de computabilidade e tratabilidade usando lógica formal no contexto de problemas computacionais. A lógica, enquanto ferramenta estruturante formal do pensamento, oferece não apenas base conceitual para a formulação de problemas, mas também mecanismos  formais para a resolução dos seus exemplos, ou a demonstração de insolubilidade. Nesse contexto, a lógica proposicional e a lógica de primeira ordem aqui são especificadas e usadas como linguagens alternativas, para descrever e manipular problemas computacionais complexos, com destaque para aplicação no problema de satisfatibilidade booleana.

Um foco particular deste estudo é o problema \sigla{MAX-SAT}{Satisfatibilidade Máxima} e sua variante parcial ponderada, que se configura como um caso representativo da fronteira entre tratabilidade e intratabilidade. Ao representar instâncias do MAX-SAT em linguagem lógica, torna-se possível aplicar técnicas formais para explorar seu comportamento sob diferentes condições. Entre essas condições, destaca-se o fenômeno da transição de fase, no qual pequenas variações estruturais nas instâncias do problema provocam mudanças abruptas entre a satisfatibilidade e a insatisfatibilidade, afetando diretamente a dificuldade computacional observada. A análise dessa transição revela aspectos fundamentais sobre o desempenho de \textit{solvers} e sobre a estrutura lógica das fórmulas envolvidas. 

A partir disso, a estrutura desta versão incompleta do trabalho contempla uma fundamentação teórica que percorre desde a lógica clássica até os formalismos necessários para a compreensão da complexidade computacional de problemas representáveis segundo aquela lógica. Em seguida, apresenta-se uma revisão bibliográfica sistemática da literatura sobre lógica e o problema MAX-SAT, com atenção ao fenômeno de transição de fase. Por fim, são discutidas observações empíricas obtidas por meio da geração controlada de exemplos, com ênfase na análise de desempenho computacional, nas fórmulas e na relevância da lógica como instrumento para investigar os limites do que é possível computar, e do que é possível computar eficientemente.